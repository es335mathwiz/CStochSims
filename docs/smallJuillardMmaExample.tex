
\subsection{Small Juillard}
\label{sec:small-juillard}
There are two example .mth files: \mmaFile{tryStack.mth} and \mmaFile{tryStoch.mth} that apply the code to a 5 equation model described in an old paper by
Michel Juillard.\footnote{I don't have the reference yet.} The examples should
be simplified and reconciled to use same inputs.  Also the inputs may need
to move to FRBUS xml format.



For example running


\begin{verbatim}
stochSim[2, 2, 1, testModel, 1, tMinusOne] // Chop // TableForm
\end{verbatim}
computes a time t-1 expectations solution using data point $t_0=t_f=2$, using a pathLength of 1 and one replication draw 
which in this particular run uses
shock {23}
\begin{gather*}\epsilon_0=
  \begin{bmatrix}\begin{array}{ccccc}
 -0.0242615 & 0.0774756 & -0.0922097 & -0.171166 & -0.121133 \\
\end{array}
  \end{bmatrix}
\end{gather*}

\begin{gather*}
  \begin{bmatrix}\begin{array}{ccccc}
 -0.3 & -0.6 & -0.3 & -0.6 & -0.3 \\
 -0.0242615 & -0.323406 & -0.33979 & -0.626224 & 0.515161 \\
\end{array}
\end{bmatrix}\end{gather*}

Running 
\begin{verbatim}
(aimType2[testModel,1,{-0.3, -0.6, -0.3, -0.6, -0.3}]//Chop)
\end{verbatim}
for horizon length 1 produces:
{\small
\begin{gather*}\begin{array}{c}
 0 \\
 -0.400792 \\
 -0.262844 \\
 -0.641487 \\
 0.560887 \\
 0 \\
 -0.0653212 \\
 0.1506 \\
 -0.0902463 \\
 0.105717 \\
\end{array}
\end{gather*}
}




producing $x_t$
\begin{gather*}\begin{array}{c}
 -0.3 \\
 -0.6 \\
 -0.3 \\
 -0.6 \\
 -0.3 \\
 -0.0242615 \\
 -0.323406 \\
 -0.33979 \\
 -0.626224 \\
 0.515161 \\
 0 \\
 -0.0653212 \\
 0.1506 \\
 -0.0902463 \\
 0.105717 \\
\end{array}
\end{gather*}



Computing t expectations 


\begin{verbatim}
stochSim[2, 2, 1, testModel, 1, t]
\end{verbatim}
computes a time t expectations solution  using data point $t_0=t_f=2$, using a pathLength of 1 and one replication draw 
which in this particular run uses
shock {4}
\begin{gather*}\begin{array}{ccccc}
 0.0795359 & -0.0172917 & -0.139301 & -0.0795452 & -0.000764437 \\
\end{array}
\end{gather*}
.
\begin{gather*}\begin{array}{ccccc}
 -0.3 & -0.6 & -0.3 & -0.6 & -0.3 \\
 0.0795359 & -0.541297 & -0.524262 & -0.887124 & 0.816313 \\
\end{array}
\end{gather*}







Running 
\begin{verbatim}
(aimType2Terror[shock,testModel,1,{-0.3, -0.6, -0.3, -0.6, -0.3}]//Chop)
\end{verbatim}
for horizon length 1 produces:





{\small
\begin{gather*}\begin{array}{c}
 0.0795359 \\
 -0.541297 \\
 -0.524262 \\
 -0.887124 \\
 0.816313 \\
 0 \\
 -0.363679 \\
 -0.211384 \\
 -0.550115 \\
 0.540922 \\
\end{array}
\end{gather*}
}

Computing convergent solutions requires a longer pathlength than 1.

{\small
\begin{gather*}\left(
\begin{array}{cc}
 -0.3 & -0.3 \\
 -0.6 & -0.6 \\
 -0.3 & -0.3 \\
 -0.6 & -0.6 \\
 -0.3 & -0.3 \\
 0.0795359 & 0.0795359 \\
 -0.609851 & -0.541297 \\
 -0.593483 & -0.524262 \\
 -1.02495 & -0.887124 \\
 0.884149 & 0.816313 \\
 0 & 0 \\
 -0.529387 & -0.363679 \\
 -0.357171 & -0.211384 \\
 -0.861942 & -0.550115 \\
 0.72622 & 0.540922 \\
\end{array}
\right)
\end{gather*}
}






For example running 


\begin{verbatim}
stochSim[2, 2, 1, testModel, 20, tMinusOne] // Chop // TableForm
\end{verbatim}
computes a time t-1 expectations solution using data point $t_0=t_f=2$, using a pathLength of 20 and one replication draw 
which in this particular run uses
shock {2}
\begin{gather*}\epsilon_0=
  \begin{bmatrix}\begin{array}{ccccc}
 0.0108933 & 0.0524847 & -0.0957718 & 0.0757908 & -0.108245 \\
\end{array}
  \end{bmatrix}
\end{gather*}

\begin{gather*}
  \begin{bmatrix}\begin{array}{ccccc}
 -0.3 & -0.6 & -0.3 & -0.6 & -0.3 \\
 0.0108933 & -0.657437 & -0.560213 & -1.15246 & 0.744063 \\
\end{array}
\end{bmatrix}\end{gather*}

Running 
\begin{verbatim}
(aimType2[testModel,1,{-0.3, -0.6, -0.3, -0.6, -0.3}]//Chop)
\end{verbatim}
for horizon length 20 produces:
{\small
\begin{gather*}\begin{array}{c}
 0 \\
 -0.70982 \\
 -0.57482 \\
 -1.26284 \\
 0.866624 \\
 0 \\
 -0.812597 \\
 -0.608628 \\
 -1.39681 \\
 1.04098 \\
\end{array}
\end{gather*}
}




producing $x_t$
\begin{gather*}\begin{array}{c}
 -0.3 \\
 -0.6 \\
 -0.3 \\
 -0.6 \\
 -0.3 \\
 0.0108933 \\
 -0.657437 \\
 -0.560213 \\
 -1.15246 \\
 0.744063 \\
 0 \\
 -0.812597 \\
 -0.608628 \\
 -1.39681 \\
 1.04098 \\
\end{array}
\end{gather*}



Computing t expectations 


\begin{verbatim}
stochSim[2, 2, 1, testModel, 20, t]
\end{verbatim}
computes a time t expectations solution  using data point $t_0=t_f=2$, using a pathLength of 1 and one replication draw 
which in this particular run uses
shock {29}
\begin{gather*}\begin{array}{ccccc}
 0.162466 & -0.0130103 & -0.130607 & -0.00796134 & 0.143759 \\
\end{array}
\end{gather*}
.
\begin{gather*}\begin{array}{ccccc}
 -0.3 & -0.6 & -0.3 & -0.6 & -0.3 \\
 0.162466 & -0.757136 & -0.626508 & -1.21833 & 1.06104 \\
\end{array}
\end{gather*}







Running 
\begin{verbatim}
(aimType2Terror[shock,testModel,20,{-0.3, -0.6, -0.3, -0.6, -0.3}]//Chop)
\end{verbatim}
for horizon length 20 produces:





{\small
\begin{gather*}\begin{array}{c}
 0.162466 \\
 -0.757136 \\
 -0.626508 \\
 -1.21833 \\
 1.06104 \\
 0 \\
 -0.895729 \\
 -0.736494 \\
 -1.60798 \\
 1.0792 \\
\end{array}
\end{gather*}
}



Solutions of pathlength 25 26 produce the same solution for $x_t$.



For example running 


\begin{verbatim}
stochSim[2, 2, 1, testModel, 25, tMinusOne] // Chop // TableForm
\end{verbatim}
computes a time t-1 expectations solution using data point $t_0=t_f=2$, using a pathLength of 25 and one replication draw 
which in this particular run uses
shock {12}
\begin{gather*}\epsilon_0=
  \begin{bmatrix}\begin{array}{ccccc}
 -0.129699 & 0.00569999 & -0.0247515 & 0.15878 & 0.0776014 \\
\end{array}
  \end{bmatrix}
\end{gather*}

\begin{gather*}
  \begin{bmatrix}\begin{array}{ccccc}
 -0.3 & -0.6 & -0.3 & -0.6 & -0.3 \\
 -0.129699 & -0.704146 & -0.45934 & -1.1226 & 0.831055 \\
\end{array}
\end{bmatrix}\end{gather*}

Running 
\begin{verbatim}
(aimType2[testModel,1,{-0.3, -0.6, -0.3, -0.6, -0.3}]//Chop)
\end{verbatim}
for horizon length 25 produces:
{\small
\begin{gather*}\begin{array}{c}
 0 \\
 -0.70982 \\
 -0.57482 \\
 -1.26284 \\
 0.866624 \\
 0 \\
 -0.812597 \\
 -0.608628 \\
 -1.39681 \\
 1.04098 \\
\end{array}
\end{gather*}
}





producing $x_t$
\begin{gather*}\begin{array}{c}
 -0.3 \\
 -0.6 \\
 -0.3 \\
 -0.6 \\
 -0.3 \\
 -0.129699 \\
 -0.704146 \\
 -0.45934 \\
 -1.1226 \\
 0.831055 \\
 0 \\
 -0.812597 \\
 -0.608628 \\
 -1.39681 \\
 1.04098 \\
\end{array}
\end{gather*}



For pathlenghts 25 and 26 we get:
{\small
\begin{gather*}\left(
\begin{array}{cc}
 -0.3 & -0.3 \\
 -0.6 & -0.6 \\
 -0.3 & -0.3 \\
 -0.6 & -0.6 \\
 -0.3 & -0.3 \\
 0 & 0 \\
 -0.70982 & -0.70982 \\
 -0.57482 & -0.57482 \\
 -1.26284 & -1.26284 \\
 0.866624 & 0.866624 \\
 0 & 0 \\
 -0.812597 & -0.812597 \\
 -0.608628 & -0.608628 \\
 -1.39681 & -1.39681 \\
 1.04098 & 1.04098 \\
\end{array}
\right)
\end{gather*}
}




Computing t expectations 


\begin{verbatim}
stochSim[2, 2, 1, testModel, 25, t]
\end{verbatim}
computes a time t expectations solution  using data point $t_0=t_f=2$, using a pathLength of 1 and one replication draw 
which in this particular run uses
shock {12}
\begin{gather*}\begin{array}{ccccc}
 -0.129699 & 0.00569999 & -0.0247515 & 0.15878 & 0.0776014 \\
\end{array}
\end{gather*}
.
\begin{gather*}\begin{array}{ccccc}
 -0.3 & -0.6 & -0.3 & -0.6 & -0.3 \\
 -0.129699 & -0.715433 & -0.470737 & -1.1453 & 0.842223 \\
\end{array}
\end{gather*}







Running 
\begin{verbatim}
(aimType2Terror[shock,testModel,25,{-0.3, -0.6, -0.3, -0.6, -0.3}]//Chop)
\end{verbatim}
for horizon length 25 produces:





{\small
\begin{gather*}\begin{array}{c}
 -0.129699 \\
 -0.715433 \\
 -0.470737 \\
 -1.1453 \\
 0.842223 \\
 0 \\
 -0.83988 \\
 -0.590985 \\
 -1.40646 \\
 1.11318 \\
\end{array}
\end{gather*}
}



For pathlenghts 25 and 26 we get:
{\small
\begin{gather*}\left(
\begin{array}{cc}
 -0.3 & -0.3 \\
 -0.6 & -0.6 \\
 -0.3 & -0.3 \\
 -0.6 & -0.6 \\
 -0.3 & -0.3 \\
 -0.129699 & -0.129699 \\
 -0.715433 & -0.715433 \\
 -0.470737 & -0.470737 \\
 -1.1453 & -1.1453 \\
 0.842223 & 0.842223 \\
 0 & 0 \\
 -0.83988 & -0.83988 \\
 -0.590985 & -0.590985 \\
 -1.40646 & -1.40646 \\
 1.11318 & 1.11318 \\
\end{array}
\right)
\end{gather*}
}