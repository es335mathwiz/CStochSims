\documentclass[12pt]{article}

\usepackage{amsmath}
\usepackage[margin=1.0in]{geometry}
\usepackage{hyperref}
\usepackage{moreverb}
\usepackage{graphicx}
\newcommand{\mmaFile}[1]{{\bf \em #1 }}
\newcommand{\nShks}{{T_\epsilon}}
\newcommand{\nXs}{{T_x}}


\title{Computing Stochastic Simulations on a Cluster}
\author{Gary S Anderson and James Trevino}
\begin{document}
\maketitle
\section{Algorithm Overview}
\label{sec:algorithm-overview}

Reconcile the generation of the Mathematica packages with the .w file. At the moment, C code generated along side .m via nuweb, but not the package in stackStochSim repository.

We are given a fixed set of $\nShks$ ``historical shocks''
\begin{gather*}
  \Upsilon =
  \begin{bmatrix}
    \epsilon_1&\cdots&\epsilon_{\nShks}
  \end{bmatrix}
\end{gather*}
and $\nXs$ ``historical data'' observations
\begin{gather*}
  X =
  \begin{bmatrix}
    x_1&\cdots&x_{T_x}
  \end{bmatrix}.
\end{gather*}
We will need to implement and describe how to accomodate FRBUS style ``baseline paths.''
\section{Example Models}
\label{sec:example-models}

There is a Mathematica implementation of the stack algorithm
 and stochastic simulations in the statckStochSims repository.  



\subsection{Small Juillard}
\label{sec:small-juillard}
There are two example .mth files: \mmaFile{tryStack.mth} and \mmaFile{tryStoch.mth} that apply the code to a 5 equation model described in an old paper by
Michel Juillard.\footnote{I don't have the reference yet.} The examples should
be simplified and reconciled to use same inputs.  Also the inputs may need
to move to FRBUS xml format.

(Should change this section to use Splice so that data is consistent.)

For example running
\begin{verbatim}
stochSim[2, 2, 1, testModel, 1, t]
\end{verbatim}
\includegraphics[width=5.0in]{../../stackStochSims/models/juillardExample/resT.pdf}

computes a time t expectation solution which in this particular run uses
shock 9.

\begin{verbatim}
stochSim[2, 2, 1, testModel, 1, tMinusOne] // Chop // TableForm
\end{verbatim}

\includegraphics[width=5.0in]{../../stackStochSims/models/juillardExample/resTMinusOne.pdf}

computes a time t-1 expectation solution which in this particular run uses
shock 3


\newpage
\section{Outstanding Questions}
\begin{description}
\item[Debugging Old C/Mathematica Code] 
\item[XML FRBUS model equation generation] 
\item[Example Models] \ 
  \begin{description}
  \item[RBC] 
  \item[Small Juillard] 
  \item[Canada] 
  \item[FRBUS] 
  \end{description}
\item[Honoring the ZLB] 
\item[Addressing ``Large Roots Problem''] 
\item[Parallel  (MPI?)] 
\item[Asymptotic Behavior -- Growth Path] 
\item[Reconcile Nonlinear FRBUS and March 3 memo Benefits of LinVer] 
\item[OMNI Rule?] 
\end{description}




\newpage
\appendix

\section{Appendix}
\label{sec:app}

\subsection{Juillard Model}
\label{sec:juillard-model-}

\subsubsection{tryStack.mth}
\label{sec:trystack.mth}


\listinginput{1}{../../stackStochSims/tryStack.mth}

\subsubsection{tryStoch.mth}
\label{sec:trystack.mth}


\listinginput{1}{../../stackStochSims/tryStoch.mth}

\subsubsection{xData}
\label{sec:xdata}

\includegraphics[width=5.0in,height=8.5in]{../../stackStochSims/models/juillardExample/xdata.pdf}

\subsubsection{shks}
\label{sec:xdata}

\includegraphics[width=5.0in,height=8.5in]{../../stackStochSims/models/juillardExample/shks.pdf}

\end{document}
