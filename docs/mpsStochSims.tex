\documentclass[12pt]{article}
\usepackage{currfile}
\RequirePackage[log]{snapshot}
\usepackage{rotating}
\usepackage{amsmath}
\usepackage{amssymb}
\usepackage{program}
\usepackage[margin=1.0in]{geometry}
\usepackage{moreverb}
\usepackage{graphicx}
\usepackage{hyperref}
\newcommand{\mmaFile}[1]{{\bf \em #1 }}
\newcommand{\nShks}{{T_\epsilon}}
\newcommand{\nXs}{{T_x}}


\title{Computing Stochastic Simulations on a Cluster ({Fast, Robust and Accurate Perfect Foresight Stochastic Simulations: Models with Growth})}
\author{Gary S Anderson and James Trevino}
\begin{document}
\maketitle


This document was generated using \currfilename.

  \begin{abstract}
This paper describes how to compute stochastic simulations on a cluster of linux machines.

Linearizing non linear models about their steady state makes it possible
to use the Anderson-Moore Algorithm(AMA) to investigate their saddle point 
properties and to efficiently compute their solutions.
Using AMA to check the long run dynamics 
of non linear  models 
avoids many of the burdensome computations associated with 
alternative methods for verifying the saddle point property.
In addition,
 for models that have the saddle point property, AMA provides a set of 
terminal conditions for solving the non linear model
that work better than the traditional approach of 
setting the end of the trajectory to the steady state values.
Furthermore, the 
asymptotic linear constraints can also generate initial conditions
for the solution path
that are better than initializing the solution path to the steady
state values.
Using the improved asymptotic constraints typically halves the computational
burden associated with solving the nonlinear problem.
  \end{abstract}

\newpage
\tableofcontents


\section{A Non-Linear Extension of the Anderson-Moore Algorithm(AMA)}

\subsection{  General Model Specification}
          Consider the model
            

\begin{gather}
          h(x_{t-\tau}  , x_{t-\tau+1}     ,..., x_{t+\theta-1}     , x_{t+\theta}   ) = 0  \label{eq:system}\\
          t = 0,\ldots, \infty\notag
\end{gather}
Where $x \in \Re^L~and~ h:\Re^{L(\tau+1+\theta)} \rightarrow \Re^L$.
We want to determine the solutions to Equation  \ref{eq:system} 
with initial conditions
          \begin{gather}
          x_i  = \bar{x}_i~ for~ i = - \tau,...,-1 \label{eq:sysInit}\\
\intertext{satisfying}
          \lim_{t \rightarrow \infty}   x_{t}  = x^\ast.\label{eq:sysLim}
\end{gather}

This paper shows how to adapt the 
methods of \cite{ANDER:AIM2} to determine the existence, and local
uniqueness of the solution to Equation \ref{eq:system}.


See also\cite{brayton11:_two_pract_algor}

\subsection{Asymptotic Linearization}
If $h$ were linear, we could immediately apply the methods of
\cite{ANDER:AIM2} to determine the existence and uniqueness of a perfect 
foresight solution and to
compute the solution.
  Since $h$ is non-linear, we will compute approximate solutions to system
system \ref{eq:system} by  using the nonlinear $h$ constraints in 
Equation \ref{eq:system}
  for the
initial part of the trajectory, and using a system
of linear constraints which reflect the asymptotic properties of the system
 for the remainder of the trajectory.

This technique can be thought of as a generalization of the approach
used by Fair-Taylor\cite{TAYLOR:CAPT}.
This paper describes a procedure which,
unlike the Fair-Taylor approach,
allows the solution to lie in the {\em stable subspace} 
of a linear
system characterizing the asymptotic properties of the
nonlinear system.


The steady state value $x^\ast$ satisfies
\begin{gather}
          h(x^\ast,...,x^\ast) = 0              \label{eq:fixpt}
\end{gather}
Near the steady state, the linear first-order Taylor expansion of h about $x^\ast$ provides a good approximation to the function h.
%xxxxxxxxx
% \begin{gather}
% h(x_{t-\tau}  ,...,x_{t+\theta})\approx
% \sum_{i=-\tau}^{\theta} \left . H_i \right  |_{x^\ast}
%  (x_{t+i} - x^\ast   ) \label{eq:asympt}
% \end{gather}
The technique presented in \cite{ANDER:AIM2} can determine the existence
and uniqueness of perfect foresight solutions near the steady state of
linear models. 
The asymptotic analysis of the linear model
determines convergence properties before burdensome calculations
of the nonlinear solutions.

That stability analysis produces a matrix, $Q$, which restricts values of the
endogenous variables to the stable subspace of the linearized system.
For trajectories which approach  a steady state,    one can  ultimately
 replace the non-linear system with the constraints codified in the matrix $Q$.

 \begin{gather}
Q
\left [
\begin{array}{c}
x_{T-\tau}-x^\ast\\
\vdots\\
x_{T}-x^\ast\\
\vdots\\
x_{T+\theta}-x^\ast\\
\end{array}
\right ]
=0
\label{eq:q}
 \end{gather}
Consequently, for solutions which converge to the steady state, 
 we can, in principal, compute solutions
to whatever accuracy required by increasing the magnitude of T.


\subsection{Relationship to Traditional Approach Using Fixed Points }
The more traditional Fair-Taylor approach also
increases T to increase accuracy, but it imposes Equation \ref{eq:ft}
\begin{gather}
I
\left [
\begin{array}{c}
x_{T+1}-x^\ast\\
\vdots\\
x_{T+\theta}-x^\ast\\
\end{array}
\right ]
=0\label{eq:ft}
\end{gather}
 instead  of equation \ref{eq:q}. 
Since Equation \ref{eq:q} more accurately characterizes the dynamics of the
nonlinear system near the steady state the approach described in this paper
converges more quickly.

It will be convenient to normalize the Q matrix so that there is the negative
of the identity matrix in the rightmost block. 
\cite{ANDER:AIM2} shows that such a normalization exists
for models which have uniquely
convergent saddle points paths from arbitrary initial conditions. 

\begin{gather}
Q^N=
\begin{bmatrix}
    -B_1&I\\-B_2&&I\\ \vdots&&&\ddots \\ -B_{\theta-1}&&&&I \\-B_\theta&&&&&I
\end{bmatrix}
\end{gather}

Thus, the traditional approach of setting the end of the trajectory to 
the steady state would be equivalent to zeroing out the left half of the
normalized Q matrix.

Using AMA to Restrict the end of the trajectory to the
asymptotic stable linear subspace provides a better approximation of the
asymptotic behavior of the non linear function.
This improvement in the approximation is reflected in the length of
the trajectory needed to achieve a given level of accuracy for the
values at the beginning of the trajectory.
In order to achieve a specific number of
significant digits in the computation of the points near the beginning
of the trajectory, setting the end of the trajectory  equal to a
specific constant would force us to  compute a longer solution path than
adopting our approach of
restricting the solution to the asymptotic linear space.

See also\cite{brayton11:_two_pract_algor}


\section{Components of the Algorithm for Computing the Convergent Path}
One can apply Newton's Algorithm to compute  the solution to the
non linear system Equations \ref{eq:system}-\ref{eq:sysLim}.
With
\begin{gather}
y_t = \label{eq:withMho}
\begin{bmatrix}
x_{t-\tau}\\ x_{t-\tau+1}\\ \vdots \\ x_{t+\theta -1} \\x_{t+\theta}
\end{bmatrix}\\
z(T) = \begin{bmatrix}
x_{-\tau}\\ x_{-\tau+1}\\ \vdots \\ x_{T+\theta -1} \\x_{T+\theta}
\end{bmatrix}\\
\intertext{Equations \ref{eq:system}-\ref{eq:sysLim} become}
\aleph(z(T),\mho)=
\begin{bmatrix}
  \begin{bmatrix}
x_{-\tau}-\bar{x}_{-\tau}\\
\vdots\\
x_{-1}-\bar{x}_{-1}
  \end{bmatrix}\\
  h(y_1)\\h(y_2)\\ \vdots\\ h(y_{T-1}) \\ h(y_T)\\
 \mho  \begin{bmatrix}
x_{T+1-\tau}-x^\ast\\
\vdots\\
x_{T+\theta}-x^\ast
  \end{bmatrix}
\end{bmatrix}
\end{gather}

Where
\begin{gather*}
  \mho=
  \begin{cases}
    Q&\text{Asymptotic Linearization}\\
    \begin{bmatrix}
      0&I
    \end{bmatrix}&\text{Traditional Fixed Point Constraint}
  \end{cases}
\end{gather*}

Figures \ref{fig:pseudoA} and  \ref{fig:pseudoB} present pseudo code
describing the algorithms for analyzing the steady state and
computing the convergent path.


\begin{figure}[htbp]
  \begin{center}
    \leavevmode
  \begin{programbox}
\BEGIN
\IF \neg |succeedsQ|(|xStar|:=|computeFixedPoint|(|h|,|xGuessFP|))\\
\THEN |fail|\COMMENT{unable to compute fixed point}
\ELSE |H|:=|linearize|(|h|,|xStar|)\\
\IF \neg |hasSaddlePointPropertyQ|(|Q|:=|andersonMoore(H)|)\\
\THEN |fail|\COMMENT{no saddle point property at this fixed point}
\ELSE
\IF \neg |hasConvergedQ|(|xPath|:=\\|convergentPath|(|xHistory|,|h|,|Q|,|T|_{MIN},|T|_{MAX}))\\
\THEN |fail|\COMMENT{path has yet to converge}
\ELSE |success|(|xPath|)\\
\FI
\FI
\FI
\END
\end{programbox}
    \caption{ Nonlinear Extension of Anderson-Moore Algorithm: Initial Setup}
    \label{fig:pseudoA}
  \end{center}
\end{figure}

\begin{figure}[htbp]
  \begin{center}
    \leavevmode
  \begin{programbox}
\BEGIN
|T|:=|T|_{MIN}\\
|xPathOld|:=|solveNonLinearSystem|(|xHistory|,|h|,|Q|,|T|_{MIN},|xGuessPath|)\\
|T|:=|T|+ \Delta |T|\\
|xPathNew|:=|solveNonLinearSystem|(|xHistory|,|h|,|Q|,|T|)\\
\WHILE (|xPathOld| \neq |xPathNew|) \wedge (|T| \leq |T|_{MAX})\DO
|xPathOld|:=|xPathNew|\\
|T|:=|T|+ \Delta |T|\\
|xPathNew|:=|solveNonLinearSystem|(|xHistory|,|h|,|Q|,|T|)\OD
\END
\end{programbox}

    \caption{ Nonlinear Extension of Anderson-Moore Algorithm: convergentPath}

    \label{fig:pseudoB}
  \end{center}
\end{figure}
\subsection{Improved Model Diagnostics}

It is possible to choose the parameters of the model and initial conditions
so that the 
number of time periods to convergence  is arbitrarily large.
Thus, for some parameter settings,
procedures  which depends on failure to converge will have trouble
determining the appropriateness of the asymptotic stability conditions.
The asymptotic linearization approach provides this information  near the
beginning of computation
before undertaking many costly computations leading to uncertain results.

\subsubsection{Computational Results}
The approach of this paper focuses computational
resources on computing saddle point paths for models
which have saddle point paths.
The analysis of the previous section indicates that the money demand
model will have convergent perfect foresight paths to the 
$s=0$ fixed point for $0 < \lambda < 1$ and to the $s=\frac{\lambda-1}{\lambda}$
for $ 1 < \lambda < 3$. There is no need to attempt solutions for models
with values of $\lambda$ outside this range.

%\input{preBou}



\subsection{Improved Initial Path Guess}
\label{sec:initGuess}

The Newton iteration requires an initial guess, $z^0(T)$.
Define
\begin{gather}
z^\ast(T^0,\mho) \ni \aleph(z^\ast(T^0,\mho),\mho)=0\label{eq:horizSoln}
\end{gather}

The $z^\ast(T^0,\mho)$ represent solutions to Equation \ref{eq:withMho}
using $T^0$ non linear time periods before applying asymptotic constraint
$\mho$.
Using iterative techniques to get a solution for  
$z^\ast(T,\mho), T> T^0$ will require an initial guess $z^0(T)$

\subsubsection{Steady State Bootstrap}

The traditional approach 
augments the shorter solution trajectory $z^0(T^0)$ with the fixed point values.

\begin{gather*}
  z^0(T)= 
  \begin{bmatrix}
z^\ast(T^0,
\begin{bmatrix}
  0&I
\end{bmatrix}
)\\
    x^\ast_{T^0+1}\\
    x^\ast_{T^0+2}\\
\vdots\\
    x^\ast_{T+\theta-1}\\
    x^\ast_{T+\theta}
  \end{bmatrix}
\end{gather*}





\subsubsection{ AMA Bootstrap}

Alternatively, one could
augment the shorter solution trajectory $z^0(T^0)$ with 
values consistent with the asymptotic behavior of the non linear
system near the fixed point.
\begin{gather*}
  z^0(T)= 
  \begin{bmatrix}
z^\ast(T^0,Q)\\
\hat{x}_{T^0+1}\\
\hat{x}_{T^0+2}\\
\vdots \\
\hat{x}_{T+\theta-1}\\
    \hat{x}_{T+\theta}
  \end{bmatrix}\\
\intertext{ with}
\hat{x}_{t}= x^\ast + 
B_1 
    \begin{bmatrix}
    (\hat{x}_{t-\tau}-x^\ast)\\
\vdots\\
    (\hat{x}_{t-1}-x^\ast)
    \end{bmatrix}  \forall t > T_0
\end{gather*}

Where $B_1$ comes from the first few rows of $Q^N$ of equation \ref{eq:qn} on
page \pageref{eq:qn}.

Appendix \ref{sec:exampleAimBoot} presents equations describing the AMA
bootstrap applied to the Money Demand Model.

\subsubsection{Computational Results}

Using $B_1$ reduces the number of Newton steps required to compute a
path of given horizon length
whether or not using Q for the asymptotic constraint.
Figure \ref{fig:newtQMon}  and \ref{fig:newtQBou} show the number of 
newton steps needed to 
move from the initial guess to the solution for each horizon length.
The line labeled ``FP Initialization'' shows the number of steps required
when setting the entire initial path guess to the fixed point values.
The line labeled ``Q Initialization'' shows the number of steps required
when setting the initial path guess to the result of applying
the $B_1$ matrix to the initial conditions given in equation \ref{eq:sysInit}.
The line labeled ``FP Extension'' shows the number of steps required
when applying the Steady State Bootstrap to the  solution from  a horizon
on period shorter.
The line labeled ``Q Extension'' shows the number of steps required
when applying the AMA Bootstrap to the  solution from  a horizon
one period shorter. 
The ``Q Extension'' and ``Q Initialization''
lines show the number of Newton steps required to solve
equation \ref{eq:horizSoln} with $\mho=Q$.
The ``FP Extension'' and ``FP Initialization''
lines show the number of Newton steps required to solve
equation \ref{eq:horizSoln} with $\mho=\begin{bmatrix}0 &I\end{bmatrix}$.
These results are  typical
for applying the two initial path guess strategies to the two models.

The AMA Bootstrap minimizes the number of Newton steps for finding the 
$z^\ast(T,\mho)$ for both models. 
Figure \ref{fig:newtQMon} presents computational results for the Money Demand
Model.
For example,
Figure \ref{fig:newtQMon} indicates
that at a horizon of 5 periods, initializing the path to the steady state
lead to 13 newton steps.
Initializing the path to the solution
obtained by applying the asymptotic linearization to the initial conditions
alone lead to 7 Newton steps. Extending the 4 period solution by adding one
period of fixed point values leads to 5 Newton steps. Using AMA to augment
the 4 period solution leads to 3 Newton Steps.

Figure \ref{fig:newtQBou} presents computational results for the 
Boucekkine Model.
Figure \ref{fig:newtQBou} indicates
that at a horizon of 7 periods, initializing the path to the steady state
lead to 5 newton steps.
Initializing the path to the solution
obtained by applying the asymptotic linearization to the initial conditions
alone lead to 4 Newton steps. Extending the 6 period solution by adding one
period of fixed point values leads to 3 Newton steps. Using AMA to augment
the 6 period solution leads to 3 Newton Steps.
Extending the path using the Fixed Point Bootstrap or the AMA Bootstrap
lead to the same number of Newton steps.
The next section will show that the AMA Bootstrap dominates since
the ``FP'' algorithms require more iterations to converge 
to the same accuracy than the ``Q'' algorithms.



\begin{figure}[htbp]
  \begin{center}
    \leavevmode
\fbox{
%%    \epsfig{file=monNewtSteps.ps}
%    cp ../../cFortran/nlaim/current/monNewtSteps.ps monNewtSteps.ps
%%    \epsfig{file=../../cFortran/nlaim/current/monNewtSteps.ps}
}    \caption{Newton Steps as Function of Horizon Length for Various Initial Guesses for Money Demand Model}
  \label{fig:newtQMon}
  \fbox{
%    \epsfig{file=bouNewtSteps.ps}
%    cp ../../cFortran/nlaim/current/bouNewtSteps.ps bouNewtSteps.ps
%%    \epsfig{file=../../cFortran/nlaim/current/bouNewtSteps.ps}
}    \caption{Newton Steps as Function of Horizon Length for Various Initial Guesses for Boucekkine Model}
  \label{fig:newtQBou}
  \end{center}
\end{figure}



\subsection{Shorter Computation Horizon for Given Computation Error}
\label{sec:shorter}

When solving models with the saddle point property
near the steady state,
the two approaches compute equivalent paths.
However, using Q
improves the tradeoff between computation horizon and solution accuracy.
For a given level of precision,
the asymptotic linearization approach obtains the solution
 with a shorter computation horizon.
At any given computation horizon, the asymptotic linearization approach
computes a more accurate solution.


This paper defines  numerical 
convergence for the algorithms
using a measure of relative error.
The algorithms terminate when
\begin{gather*}
  \| D^T (x - \hat{x})\| \le m \| D^T  \hat{x} \|
\end{gather*}
where 
%\input{../../cFortran/nlaim/current/wtsInfo}
% cp ../../cFortran/nlaim/current/wtsInfo.tex wtsInfo.tex}
%\input{wtsInfo}
\begin{gather*}
  \| \epsilon \| = \sqrt{ (\epsilon^T \epsilon)}\\
x-\hat{x} = S_{1k} z^k(T) - S_{2k} z^{k-1}(T)\\
\text{ with 
$ S_{1k} = 
  \begin{bmatrix}
    I_{nL}&0
  \end{bmatrix}
$ and 
$\  S_{2k} = 
  \begin{bmatrix}
    I_{nL}&0
  \end{bmatrix}
$}\\
\text{  chosen to select comparable parts of the state vector. }
\end{gather*}
If $m= 10^{-k}$ then larger components of $D x$ have k significant digits\cite{nag95}. 
The numerical calculation for this paper set $k \approx 8$.





\section{Further Extensions to the Algorithm}
\label{sec:algorithm-overview}

\begin{description}
\item[qMat] 
\item[fp] 
\item[drvFunc] 
\item[func] 
\item[xData] 
\item[shocks] 
\item[eqns] 
\item[leads] 
\item[lags] 
\end{description}



Reconcile the generation of the Mathematica packages with the .w file. At the moment, C code generated along side .m via nuweb, but not the package in stackStochSim repository.

We are given a fixed set of $\nShks$ ``historical shocks''
\begin{gather*}
  \Upsilon =
  \begin{bmatrix}
    \epsilon_1&\cdots&\epsilon_{\nShks}
  \end{bmatrix}
\end{gather*}
and $\nXs$ ``historical data'' observations
\begin{gather*}
  X =
  \begin{bmatrix}
    x_1&\cdots&x_{T_x}
  \end{bmatrix}.
\end{gather*}
We will need to implement and describe how to accomodate FRBUS style ``baseline paths.''

\subsection{Honoring the Zero Lower Bound on Interest Rates}
\label{sec:honoring-zero-lower}

\subsection{Characterizing Asymptotic Behavior:  Accomodating Growth}
\label{sec:char-asympt-behav}

\begin{description}
\item[section homepage] \ 
  \begin{itemize}
  \item \href{/mq/home/develop/frbus/models/develop/linver/aareadme}{linver generation }
    \begin{itemize}
    \item linver more variables because about 150 added to make version with only one lead one lag
    \item growth comes from price level, productivity, population but still 15 unit roots.  Flint identified source of all but  5 some vars just identities for first differences
    \item some mods to frbus to make the model more reasonable economics linver ``not exactly'' the same model perhaps it is better while generally consiste nt
    \item eliminated variables that switch signs used gross specification instead of net for example inventories and defecits
    \item Flint says linver memo can be found in rsma electronic library
    \item simplified rffe equation nonlinearities like thresholds
    \item all trending variables are converted to logs but after dynare linerization and before producing linver
    \end{itemize}

\item \href{/mq/home/develop/frbus/models/develop/linver/mcapwp/linver_mcapwp_eviews_mod.txt}{dynare linver}
  \item \href{http://www.rsma.frb.gov/cgi-bin/mq/maqswin/varinfo_by_attribute.pl}{vars in stdver aim\_translation\_rule group}
  \item \href{http://www.rsma.frb.gov/cgi-bin/mq/maqswin/stoch_sim_eviews.pl}{stochastic simulations generation page}
  \item \href{https://www.rsma.frb.gov/twiki/bin/view/RSMAQS/ExtendedExtension}{extended extension}
  \end{itemize}
\item[info in frb.xml] \ 
  \begin{itemize}
  \item 
\begin{verbatim}
  <variable>
    <name>hgx</name>
    <equation_type>Identity</equation_type>
    <sector>Aggregate Output Identities</sector>
    <definition>Trend growth rate of XG, cw 2009$ (annual rate)</definition>
    <description>Some of the components of potential output are modeled as
stochastic trends with stochastic drift terms.  Because both log level
shocks and growth rate shocks occur in this framework, the trend rate
of growth of such a component differs from the growth rate of its
trend level.  The trend rate of growth of nonfarm business output plus
oil imports is calculated using the distinct trend rates of growth,
rather than on the growth rate of trends, of multifactor productivity
(HMFPT) and those factor inputs that have trend growth rates (HLEPT,
HQLWW).  The imposed coefficients are the same as those in the
production function for the level of XGPOT.  The trend growth rate of
adjusted business sector output (HGX) is smoother than the growth rate of the
level of potential adjusted business sector output (400*del(log(XGX))).
</description>

\end{verbatim}
  \item 
\begin{verbatim}
  <variable>
    <name>lef</name>
    <equation_type>Identity</equation_type>
    <sector>Labor Market</sector>
    <definition>Federal civilian employment ex. gov. enterprise</definition>
    <description>The ratio of federal employment to constant-dollar federal
government expenditures on employee compensation (EGFL) depends on the
exogeous proportionality factor ULEF and an adjustment for trend
productivity.  Because the national accounts assume that there is no
productivity growth in the government sector, the dummy variable
DGLPRD is set to 0 over history.  In long-run simulations, however,
DGLPRD is set to 1.0 to ensure that the government shares of
employment and GDP are stationary.  The equation is written in first
differences so that the productivity adjustment enters as a growth
rate rather than as a level.  The appropriate value of the latter
would be difficult to estimate.

</description>

\end{verbatim}
  \item 
\begin{verbatim}

The dummmy variable DMPSTB is sometimes set equal to one when the
model is used to run stochastic simulations.  This makes HLEPT equal
to the growth rate of population, a simplification that improves the
long-run stability of the model.

\end{verbatim}
  \item 
\begin{verbatim}
    <description>The target inventory-sales ratio is estimated along with the
equation for the stock of inventories (KI) using the Kalman filter.
To ensure that long-run simulations are characterized by balanced
growth, exogenous variable DGLPRD can be set to 1.0 to make the target
inventory-sales ratio constant.

\end{verbatim}
  \end{itemize}
\end{description}


\subsection{Optimal Control}
\label{sec:optimalcontrol}

\subsection{EViews Modes and Switches}

\label{sec:eviews}
\begin{itemize}
\item templates for dates and database inputs
\item inputs fame databases, plus models, eviews switches
\item tim's perl scripts and the input to the scripts.  what changes? how often? what's fixed?
\item outputs csv files?  plug into current reporting code
\item output percentiles, conditional probabilities of return to ZLB
\item James working out range of ``inputs'' and ``switches'' 
\item  versus essential
\item stochastic sim with inertial taylor rule would prove very useful
\end{itemize}


\subsection{Addressing the ``Too Many Large Roots'' Problem}
\label{sec:addressing-too-many}

\subsection{Validation}
\label{sec:validation}

Validation within the section for DSGE models.

Validation of FRBUS may require a wider engagement over a longer time horizon. 
Shoot for beginning late April early May.  JP, Hess, David R, Flint.

Depths of recessions and the types of useful shocks:  Bob, David R

\subsection{Using the Cluster:  Parallel Programming Considerations}
\label{sec:using-clust-parall}





\section{Example Models}
\label{sec:example-models}

There is a Mathematica implementation of the stack algorithm
 and stochastic simulations in the statckStochSims repository.  

\subsection{Small Juillard}
\label{sec:small-juillard}
There are two example .mth files: \mmaFile{tryStack.mth} and \mmaFile{tryStoch.mth} that apply the code to a 5 equation model described in an old paper by
Michel Juillard.\footnote{I don't have the reference yet.} The examples should
be simplified and reconciled to use same inputs.  Also the inputs may need
to move to FRBUS xml format.



For example running
%Null
\begin{verbatim}
stochSim[2, 2, 1, testModel, 1, t]
\end{verbatim}
computes a time t expectation solution which in this particular run uses
shock {14}
\begin{gather*}\begin{array}{ccccc}
 -0.135773 & 0.00968129 & 0.219991 & -0.00656889 & 0.0680888 \\
\end{array}
\end{gather*}
.
\begin{gather*}\begin{array}{ccccc}
 -0.3 & -0.6 & -0.3 & -0.6 & -0.3 \\
 -0.135773 & -0.50341 & -0.21941 & -0.930387 & 0.586411 \\
\end{array}
\end{gather*}

%Null
\begin{verbatim}
stochSim[2, 2, 1, testModel, 1, tMinusOne] // Chop // TableForm
\end{verbatim}
computes a time t-1 expectation solution which in this particular run uses
shock {3}
\begin{gather*}\begin{array}{ccccc}
 0.0794213 & 0.029234 & -0.0737597 & -0.0811142 & 0.150862 \\
\end{array}
\end{gather*}

\begin{gather*}\begin{array}{ccccc}
 -0.3 & -0.6 & -0.3 & -0.6 & -0.3 \\
 0.0794213 & -0.371396 & -0.220332 & -0.525215 & 0.670087 \\
\end{array}
\end{gather*}
\newpage

\subsection{RBC}
\label{sec:rbc}


  See for example\cite{Maliar2005}

\begin{gather*}
   \max\left \{  u(c_t^t) + E_t \sum_{\tau=t}^\infty \beta \delta^{\tau+1-t}u(c_{\tau+1}^t)\right \}\\
c_\tau^t + k_\tau^{t+1}=(1-d)k_\tau^{t-1} + \theta_\tau f(k_\tau^{t-1})\\
f(k_\tau^{t-1})= k_\tau^\alpha
 \end{gather*}

 \begin{gather*}
   paramSubs={
alpha->36/100,
delta->95/100,
rho->95/100,
sigma->1/100
} ;
 \end{gather*}


\subsection{A Money Demand Model}
          Consider the three equation non-linear system
\begin{gather}
          \ln \frac{m_t}{p_t} = \alpha  + \beta  \ln(\rho+ (\frac{p_{t+1}-p_t}{p_t})) \label{eq:mdem}\\
          m_{t}  - m_{t-1}    = \gamma (m_{t-1}    - \mu)+ \delta s_t\label{eq:msup}\\
s_t=\lambda s_{t-1}(1- s_{t-1})\label{eq:nlin}
\end{gather}
Where $L=3,\tau=1,\theta=1$, and $0 \leq \lambda, ~\alpha  < 0~,~\beta< 0~,~\rho> 0,~\gamma <0~,$ and  $\mu > 0  $ exogenously  given. 

This example augments a simple forward looking money demand function(Equation \ref{eq:mdem}) and
a money supply rule(Equation \ref{eq:msup}) with an easy to manipulate and much
studied  nonlinear function, the quadratic map(Equation \ref{eq:nlin}).
Including the quadratic map
 provides a convenient way to study the impact of model parameters on
asymptotic behavior.
The parameter $\lambda$ in the 
quadratic map provides  a simple nonlinear function that can generate
fixed points, limit cycles, and chaotic invariant sets, but
this paper will study values of $\lambda$ associated with fixed points.


%\subsection{The Money Demand Model: Dynamics Near the Fixed Points}
The points $m^\ast=\mu-\frac{- \delta s^\ast}{\gamma}~
p^\ast=m^\ast\exp^{-(\alpha+\beta\ln(\rho))},$  where 
$ s^\ast = 0\text{ or }s^\ast=\frac{\lambda-1}{\lambda}$, are
fixed points for the system.
We can linearize the system and
 investigate the dynamics of the system near either steady state.

 We want to investigate
the model with 
initial conditions

\begin{gather*}
          m_0   = \bar{m}_0
\end{gather*}
\begin{gather*}
          p_0   = \bar{p}_0
\end{gather*}
\begin{gather*}
          s_0   = \bar{s}_0
\end{gather*}
and terminal conditions
\begin{gather*}
          \lim_{t\rightarrow\infty}
\left [\begin{array}{c}m_t\\ p_t\\ s_t \end{array} \right ]=
\left [\begin{array}{c}m^\ast\\ p^\ast\\ s^\ast \end{array} \right ]
\end{gather*}




Applying the methods of \cite{ANDER:AIM2} near the fixed point, the state space transition matrix is
\begin{gather*}
A=\left [
\begin{array}{ccc}
               (1+ \gamma )  &       0&-\delta \lambda (2 s^\ast - 1)\\
   \frac{\rho / \beta}{m^\ast /p^\ast}&\frac{\beta - \rho}{\beta}&0\\
0&0&-\lambda (2s^\ast -1)\\
\end{array}
\right ]
\end{gather*}
Which has three non zero eigenvalues, $(1+\gamma)$, $\lambda (1 - 2 s^\ast)$, and  $\frac{\beta - \rho}{\beta}$.

 The first two eigenvalues are smaller than one in magnitude provided:
%xxxxxxxx
%  \begin{gather*}
% -2 < \gamma < 0 \\
% \intertext{ and }
%      s^\ast =
% \begin{cases}
% 0 \text{ and } |\lambda| < 1 \\
% \frac{\lambda - 1}{\lambda} \text{ and } 1<\lambda < 3
% \end{cases}
%  \end{gather*}
 The last eigenvalue has magnitude bigger than one since $\frac{\rho}{\beta}<0$.
The Q matrix of constraints imposing the saddle point property
 consists of two auxiliary initial conditions and one unstable left eigenvector
if  $0<\lambda<1 \text{ or } 1 < \lambda < 3$ 
\begin{gather*}
Q=
\left [
\begin{array}{cccccc}
                 -(1+ \gamma )&  0 &0&1  &     0&-\delta\\
0&0&\lambda ( 2 s^\ast -1)&0&0&1\\
0&0&0&-\frac{(-\beta + \beta \lambda + \rho - 2 \beta \lambda s^\ast)}
{(\beta \delta \lambda(1 - s^\ast))}&
-\frac{m^\ast(\beta \gamma+\rho)(-\beta + \beta \lambda + \rho - 2 \beta \lambda s^\ast)}
{\rho p^\ast (\beta \delta \lambda(1 - s^\ast))}&
1
%                   0    & 0&0  &\frac{-\rho / \beta}{m^\ast / p^\ast} &\left ( \frac{(\beta -\rho)}{\beta}+(1+\gamma)\right )&\frac{-\delta\frac{\rho / \beta}{m^\ast/p^\ast}}{(\frac{\beta - \rho}{\beta}+\delta)}\\
\end{array}
\right ]
\end{gather*}

\begin{gather*}\label{eq:qn}
Q^N=
\begin{bmatrix}
1 + \gamma & 0&\delta  \lambda  - 2 \delta  \lambda  s^\ast&-1&0&0\\
  {{\left( 1 + \gamma  \right)  \rho  p^\ast}\over 
    {\left( \beta  \gamma  + \rho  \right)  m^\ast}}&0&
  {{\delta  \lambda  \rho  \left( -\beta  + \rho  \right)  p^\ast 
      \left( -1 + 2 s^\ast \right) }\over 
    {\left( \beta  \gamma  + \rho  \right)  m^\ast 
      \left( \beta  - \beta  \lambda  - \rho  + 2 \beta  \lambda  s^\ast \right)
      }}&0&-1&0\\
0&0&\lambda  - 2 \lambda  s^\ast&0&0&-1 
\end{bmatrix} =
\begin{bmatrix}
  B_1&-I
\end{bmatrix}
\end{gather*}

\subsection{Boucekkine's Non Linear Example}
Boucekkine's\cite{BOUCEKKINE:95} presents the following
example nonlinear model. For $t>0$


  \begin{gather*}
z_t - 5 y_{1,t}^{0.15} x_{1,t}^{0.75} =0\\
0.15 \frac{y_{1,t+1}}{y_{1,t}} + 5 x_{1,t}^a - 0.25 = 0\\
y_{2,t+1} - 3 \frac{y_{2,t}^{1.65}}{x_{1,t-1}} w_{t-3}=0\\
x_{2,t} - 0.75 \frac{y_{1,t-1}}{y_{2,t}} + 1.25 = 0\\
y_{1,t+1}^b - c \frac{x_{2,t-1}}{y_{2,t+1}} y_{1,t}=0\\
w_{t} = 1
  \end{gather*}



%\input{bouceTable}



%\input{boucExampleNarr}

Appendix \ref{transMat} provides additional detail describing the
transition matrix and the auxiliary initial conditions.

%\input{transMat}

\subsection{Smets-Wouter}
\label{sec:edo}

\subsection{EDO}
\label{sec:edo}

\subsection{Canada}
\label{sec:canada}


\subsection{FRBUS (stdver)}
\label{sec:frbus-stdver}

\subsection{FRBUS (linver)}
\label{sec:frbus-linver}

\subsection{stdver  linver Comparison}
\label{sec:comparison-linver}


\section{Outstanding Questions}
\begin{description}
\item[Debugging Old C/Mathematica Code] 
\item[XML FRBUS model equation generation] 
\item[Example Models] \ 
  \begin{description}
  \item[RBC] 
  \item[Small Juillard] 
  \item[Canada] 
  \item[FRBUS] 
  \end{description}
\item[Honoring the ZLB] 
\item[Addressing ``Large Roots Problem''] 
\item[Parallel  (MPI?)] 
\item[Asymptotic Behavior -- Growth Path] 
\item[Reconcile Nonlinear FRBUS and March 3 memo Benefits of LinVer] 
\item[OMNI Rule?] 
\item[Dynare versus XML input] 
\item[check contents] \ 
  \begin{itemize}
  \item 
\begin{verbatim}
~/git/safePaperProduction/parallelFRBUS/docs/parallelize.tex
\end{verbatim}
  \item presentation of Alt Sims DSGE suite seminar (notability)
  \item frb-us overview oct 2015 presentation (notability)
  \item /msu/res2/m1gsa00/aim/frbus/ssComp.mtex
  \item /msu/res2/m1gsa00/aim/nlaimExperiments/
\end{itemize}
\end{description}





\section{Conclusions}
Linearizing non linear models about their steady state makes it possible
to use the Anderson-Moore Algorithm(AMA) to investigate their saddle point 
properties and to efficiently compute their solutions.
Using AMA to check the long run dynamics 
of non linear  models 
avoids many of the burdensome computations associated with 
alternative methods for verifying the saddle point property.
In addition,
 for models that have the saddle point property, AMA provides a set of 
terminal conditions for solving the non linear model
that work better than the traditional approach of 
setting the end of the trajectory to the steady state values.
Furthermore, the 
asymptotic linear constraints can also generate initial conditions
for the solution path
that are better than initializing the solution path to the steady
state values.
Using the improved asymptotic constraints typically halves the computational
burden associated with solving the nonlinear problem.



\nocite{GUCKENHEIMERHOLMES,FUHRER:96}
%\bibliographystyle{authordate1}
\bibliographystyle{plainnat}
\bibliography{files,anderson,second,usingAimBetter}

\newpage
\appendix

\section{command line flags}
\label{sec:command-line-flags}

\begin{description}
\item[p] 
\item[v] 
\item[V] 
\item[l] 
\item[r] 
\item[a] 
\item[s] 
\item[N] 
\item[F] 
\item[X] 
\item[H] 
\item[D] 
\item[K] 
\item[E] 
\item[L] 
\item[S] 
\item[I] 
\item[J] 
\item[h] 
\end{description}

\section{AMA Bootstrap Example}
\label{sec:contExample}
\label{sec:exampleAimBoot}


For $T^0=0$ we can use Equation \ref{eq:qn} to compute $\hat{x}_0,\hat{x}_1\ldots$ for arbitrary
initial conditions:
\begin{gather*}
\hat{x}_0=B_1
\begin{bmatrix}
  (\bar{m}_{-1}-x^\ast)\\ (\bar{p}_{-1}-p^\ast)\\ (\bar{s}_{-1}-s^\ast)
\end{bmatrix} = 
\begin{bmatrix}
  (1 + \gamma)(\bar{m}_{-1}-m^\ast) + \delta (\bar{s}_{-1}-s^\ast) \lambda (1 - 2 s^\ast)\\
\frac{(1 + \gamma)(\bar{m}_{-1}-m^\ast) p^\ast \rho}{m^\ast(\beta \gamma + \rho)} + 
\frac{\delta(\bar{s}_{-1}-s^\ast)\lambda (1 - 2 s^\ast)p^\ast \rho(\rho - \beta) }{m^\ast(\beta\gamma + \rho)(\beta - \beta\lambda - \rho + 2 \beta\lambda s^\ast)}\\
(\bar{s}_{-1}-s^\ast) \lambda (1 - 2 s^\ast)
\end{bmatrix}\\
\hat{x}_1=B 
\begin{bmatrix}
  \hat{m}_{0}\\ \hat{p}_{0}\\ \hat{s}_{0}
\end{bmatrix} = 
\begin{bmatrix}
  (1 + \gamma)(\hat{m}_{0}-m^\ast) + \delta (\hat{s}_{0}-s^\ast) \lambda (1 - 2 s^\ast)\\
\frac{(1 + \gamma)(\hat{m}_{0}-m^\ast) p^\ast \rho}{m^\ast(\beta \gamma + \rho)} + 
\frac{\delta(\hat{s}_{0}-s^\ast)\lambda (1 - 2 s^\ast)p^\ast \rho(\rho - \beta) }{m^\ast(\beta\gamma + \rho)(\beta - \beta\lambda - \rho + 2 \beta\lambda s^\ast)}\\
(\hat{s}_{0}-s^\ast) \lambda (1 - 2 s^\ast)
\end{bmatrix}\\ \vdots
\end{gather*}

Since the eigenvalues of $B_{-1}$ are $(1 + \gamma)$ and $\lambda(1- 2 s^\ast)$
the bootstrap path ultimately converges to the steady state.
The bootstrap path  approximation to the non linear solution improves 
as the solution approaches  the
steady state.

For $T^0=1$ after substituting the initial conditions and the
AMA bootstrap path, we must find $m_0,p_0,s_0$ satisfying the system
\begin{gather*}
-\alpha + \log(\frac{m_0}{p_0}) - \beta \log(\rho + \frac{\frac{(1+\gamma)p^\ast \rho m_0}{m^\ast(\beta \gamma + \rho)}- p_0 +\frac{\delta\lambda p^\ast \rho(\rho - \beta)(2 s^\ast -1)s_0}{m^\ast(\beta \gamma + \rho)(\beta - \beta \lambda -\rho + 2 \beta \lambda s^\ast)}}{p_0})=0\\
  -\bar{m}_{-1} - \gamma \bar{m}_{-1} + \gamma \mu + m_0 - \delta s_0 = 0\\
-(\bar{s}_{-1} \lambda) + \bar{s}_{-1}  \lambda + s_0 = 0
\end{gather*}
\section{Transition Matrix Details}
\label{transMat}
%\input{transMat}
\section{Mathematica Code for Models}






\subsection{Juillard Model}
\label{sec:juillard-model-}

\subsubsection{tryStack.mth}
\label{sec:trystack.mth}


\listinginput{1}{../../stackStochSims/tryStack.mth}

\subsubsection{tryStoch.mth}
\label{sec:trystack.mth}


\listinginput{1}{../../stackStochSims/tryStoch.mth}

\subsubsection{xData}
\label{sec:xdata}

\includegraphics[width=5.0in,height=8.5in]{../../stackStochSims/models/juillardExample/xdata.pdf}

\subsubsection{shks}
\label{sec:xdata}

\includegraphics[width=5.0in,height=8.5in]{../../stackStochSims/models/juillardExample/shks.pdf}

\end{document}

% \subsubsection{Computational Results}

% Table \ref{tab:comp} 
% presents some computational results for the Money Demand Model.
% The last column demonstrates
% the equivalence between the convergent solution paths
% obtained by using the asymptotic linearization and those obtained
% using the traditional fixed point constraint.
% The computations using Q and using FP each used a convergence tolerance of 
% $10^-8$.
% The $\| \cdot \|_2$ difference between the initial portions of
% the trajectories are also within the supplied convergence tolerance of 
% $10^{-8}$.

% \begin{table}[htbp]
%   \begin{center}
%     \leavevmode
% %\input{../../cFortran/nlaim/current/compareFinal}
% % cp ../../cFortran/nlaim/current/compareFinal.tex compareFinal.tex}
% \input{compareFinal}
%     \caption{Asymptotic Linearization and Fixed Point Solution Characteristics}
%     \label{tab:comp}
%   \end{center}
% \end{table}


% The following figures employ two measures of error to characterize
% the rate of convergence of the two algorithms: approximation error and 
% change in path error.
% The approximation error is
% $\| D(S_{1k}(T)x_{now} - S_{2k}(T^\ast)x_{converged})\|_2$. 
% The change in path error is
% $\|D( S_{1k}(T+1)x_{next} - S_{2k}(T)x_{now})\|_2$. 



% The top half of 
% Figure \ref{fig:aggoneandhalf} reports the variation in approximation 
% error as a function of the computation horizon when $\lambda=1.5$.
% For any given horizon, the 
% approximation error is always about 6 times larger
% when using the fixed point instead of the asymptotic
% linearization.
% The bottom half of 
% Figure \ref{fig:aggoneandhalf} emphasizes this point by
% reporting the variation in approximation 
% rescaled so that the initial approximation errors are the same.



% \begin{figure}[htbp]
%   \begin{center}
%     \leavevmode
% \fbox{
% %%    \epsfig{file=togMonFir.ps}
% %    cp ../../cFortran/nlaim/current/togMonFir.ps togMonFir.ps
% %%    \epsfig{file=../../cFortran/nlaim/current/togMonFir.ps}
% }    \caption{Approximation Error as Function of $T$ for $\lambda=1.5$}
%     \label{fig:aggoneandhalf}
% \fbox{
% %    \epsfig{file=togMonChgFir.ps}
% %    cp ../../cFortran/nlaim/current/togMonChgFir.ps togMonChgFir.ps
% %%    \epsfig{file=../../cFortran/nlaim/current/togMonChgFir.ps}
% }    \caption{Change in Path Error as Function of $T$ for $\lambda=1.5$}
%     \label{fig:deloneandhalf}
%   \end{center}
% \end{figure}

% Without prior knowledge of the convergent solution,
% algorithms rely on the change in the solution path to determine convergence.
% Figure \ref{fig:deloneandhalf} reports the variation in the change in
% path
% error as a function of the computation horizon when $\lambda=1.5$.
% The asymptotic linearization algorithm would signal convergence before
% the fixed point algorithm.
% The accuracy of the solution does not suffer since the aggregate errors are
% so much less for any given computation horizon.


% Figure \ref{fig:aggsmlposroot} reports the variation in approximation 
% error as a function of the computation horizon  when $\lambda=1.9$.
% Again, for any given horizon, the 
% approximation error is always significantly less when using the asymptotic
% linearization.
% \begin{figure}[htbp]
%   \begin{center}
%     \leavevmode
% \fbox{
% %    \epsfig{file=togMon.ps}
% %    cp ../../cFortran/nlaim/current/togMon.ps togMon.ps
% %%    \epsfig{file=../../cFortran/nlaim/current/togMon.ps}
% }    \caption{Approximation Error as Function of $T$ for $\lambda=1.9$}
%     \label{fig:aggsmlposroot}
% \fbox{
% %    \epsfig{file=togMonChg.ps}
% %    cp ../../cFortran/nlaim/current/togMonChg.ps togMonChg.ps
% %%    \epsfig{file=../../cFortran/nlaim/current/togMonChg.ps}
% }    \caption{Change in Path Error as Function of $T$ for $\lambda=1.9$}
%     \label{fig:delsmlposroot}
%   \end{center}
% \end{figure}
% Figure \ref{fig:delsmlposroot} reports the variation in the change in
% approximation 
% error as a function of the computation horizon when $\lambda=1.9$.
% The asymptotic linearization algorithm would signal convergence before
% the fixed point algorithm.



% Figure \ref{fig:aggsmlnegroot} reports the variation in approximation 
% error as a function of the computation horizon when $\lambda=2.3$.
% For any given horizon, the 
% approximation error is always significantly less when using the asymptotic
% linearization.
% \begin{figure}[htbp]
%   \begin{center}
%     \leavevmode
% \fbox{
% %    \epsfig{file=togMonThi.ps}
% %    cp ../../cFortran/nlaim/current/togMonThi.ps togMonThi.ps
% %%    \epsfig{file=../../cFortran/nlaim/current/togMonThi.ps}
% }    \caption{Approximation Error as Function of $T$ for $\lambda=2.3$}
%     \label{fig:aggsmlnegroot}
% \fbox{
% %    \epsfig{file=togMonChgThi.ps}
% %    cp ../../cFortran/nlaim/current/togMonChgThi.ps togMonChgThi.ps
% %%    \epsfig{file=../../cFortran/nlaim/current/togMonChgThi.ps}
% }    \caption{Change in Path Error as Function of $T$ for $\lambda=2.3$}
%     \label{fig:delsmlnegroot}
%   \end{center}
% \end{figure}
% Figure \ref{fig:delsmlnegroot} reports the variation in the change in
% approximation 
% error as a function of the computation horizon when $\lambda=2.3$.
% The asymptotic linearization algorithm would signal convergence before
% the fixed point algorithm.





% Figure \ref{fig:numbIntMPS} presents a density plot comparing the
% number of horizons required for convergence for the two algorithms as a
% function of the initial conditions.
% Since $m_0$ and $s_0$ depend only on initial conditions they will
% not vary as the horizon length, $T_0$, changes, but
% $p_0$ will depend on the future values and the terminal conditions and will
% vary with the horizon length.
% The asymptotic linearization converges faster than the fixed point for all
% initial conditions.
% \begin{sidewaysfigure}[htbp]
%   \begin{center}
%     \leavevmode

% \fbox{
% %    \epsfig{file=densityGraph.ps,width=5.0in}
% %    cp ../../cFortran/nlaim/current/densityGraph.ps densityGraph.ps
% %%    \epsfig{file=../../cFortran/nlaim/current/densityGraph.ps,width=5.0in}
% }    
%     \caption{Number of Iterations as a Function of Initial Conditions}
%     \label{fig:numbIntMPS}
%   \end{center}
% \end{sidewaysfigure}

% Figures \ref{fig:aggboubas} -\ref{fig:bouHorToSS} 
% present some computational results for the Boucekkine Model.
% Figure    \ref{fig:aggboubas} reports the approximation error while 
% Figure \ref{fig:delboubas} reports the variation in the change in
% approximation 
% error as a function of the computation horizon when $d=0.5$
% For any given horizon, the 
% approximation error is always significantly less when using the asymptotic
% linearization.
% The asymptotic linearization algorithm would signal convergence before
% the fixed point algorithm.
% The accuracy of the solution does not suffer since the aggregate errors are
% so much less for any given computation horizon.
% \begin{figure}[htbp]
%   \begin{center}
%     \leavevmode
% \fbox{
% %    \epsfig{file=togBou.ps}
% %    cp ../../cFortran/nlaim/current/togBou.ps togBou.ps
% %%    \epsfig{file=../../cFortran/nlaim/current/togBou.ps}
% }    \caption{Variation in Approximation Error as Function of $T$ for $a=-3,b=\frac{3}{2},c=\frac{5}{2},d=\frac{1}{2}$}
%     \label{fig:aggboubas}
% \fbox{
% %    \epsfig{file=togBouChg.ps}
% %    cp ../../cFortran/nlaim/current/togBouChg.ps togBouChg.ps
% %%    \epsfig{file=../../cFortran/nlaim/current/togBouChg.ps}
% }    \caption{Change in Path Error as Function of $T$ for $a=-3,b=\frac{3}{2},c=\frac{5}{2},d=\frac{1}{2}$}
%     \label{fig:delboubas}
%   \end{center}
% \end{figure}






% Figure \ref{fig:bouHorToSS} presents a graph  comparing the
% number of horizons required for convergence for the two algorithms as a
% function of the initial conditions for $w$.
% The asymptotic linearization converges faster than the fixed point for all
% initial conditions.
% \begin{sidewaysfigure}[htbp]
%   \begin{center}
%     \leavevmode
% \fbox{
% %    \epsfig{file=wCurve.ps,width=5.0in}    
% %    cp ../../cFortran/nlaim/current/wCurve.ps wCurve.ps    
% %%    \epsfig{file=../../cFortran/nlaim/current/wCurve.ps,width=5.0in}    
% }    \caption{Horizon Length Versus Distance to Steady State}
%     \label{fig:bouHorToSS}
%   \end{center}
% \end{sidewaysfigure}





